\begin{document}
	\newpage % Start content on a new page
	\begin{center}
		\textbf{\large{Household Vulnerability And Environmental\\ Dependence in Rural Nepal \\ \vspace{0.5cm}
				Sanjeev Nhemhafuki, Bipin Khadka, Resham Thapa }}
	\end{center}
	\begin{center}
		\section*{\large{ABSTRACT}}
	\end{center}	
	
	\addcontentsline{toc}{section}{\MakeUppercase{Abstract}}
	\renewcommand{\thepage}{\roman{page}}
	\setcounter{page}{5}
	\setstretch{1.5}
	This thesis examines whether the households dependent on environmental income are vulnerable in rural setting of three distinct geographic region of Nepal. For this purpose, the study develops a composite household vulnerability index based on the various capitals owned by the households and relates the latter with the share of environmental income to total income. The study uses the environmental augmented household-level livelihood longitudinal data-set of Nepal, known as PEN (Poverty Environment Network) dateset. It covers the period of 2006, 2009 and 2012. Further, we assess the relationship between the household vulnerability and environmental dependence. The results suggest that Environmental dependence and Household Vulnerability are positively associated. The level of vulnerability is heterogeneous in different ecological zone. Mountainous region seems more vulnerable as compared to Lowland and Mid-hill region. Rural households are confined to rural activities which are related to environment. Therefore, more environmental dependence has to do with more vulnerability. This provides critical evidence for a policy debate within society, which argues for the global implementation of Community Forest Management and Conservation Area Management.   
	
	