\section{Conclusion} \label{isect5}
\subsection{Summary of the Findings}
This paper focuses on the analysis of household vulnerability in rural Nepal, examining the inter-temporal dynamics of vulnerability and constructing a household vulnerability index (HVI). The index is developed by integrating various capitals, including human, physical, financial, livelihood strategies, and social capital. The research uses a unique and environmental augmented household-level livelihood panel data-set from 2006 to 2012, sourced from Tribhuvan University’s Institute of Forestry and the University of Copenhagen’s Department of Food and Resource Economics \citep{walelign2022unique}.

The study uses Mini-max and Maxi-min methods to construct the HVI, allowing for a comprehensive assessment of vulnerability across multiple dimensions. The analysis reveals distinct patterns, with Chitwan showing slightly variable vulnerability levels and Kaski showing stable vulnerability. Mustang indicates higher vulnerability in 2006, a slight decrease in 2009, and stability in 2012. Radar charts further depict the variability within districts. Sankey diagram helps in visualizing the persistence and transience of the levels of vulnerability (High, Moderate, Low). 

The study employs Panel Data Regression techniques to identify and quantify factors influencing household vulnerability. Environmental dependence is identified as a key contributor to increased vulnerability in rural households. A positive association between dependency ratios and the HVI is found, indicating that a higher dependency ratio exacerbates vulnerability. Additionally, shocks to households also play a significant role in elevating vulnerability levels. The research also finds a mitigating effect of debt on household vulnerability, suggesting that households with certain debt levels exhibit lower levels of vulnerability. 

\subsection{Conclusion}
The research examines the relationship between the household vulnerability and environmental dependence in the rural households across the physio-graphic regions utilizing various capital such as Human Capital, Physical Capital, Social Capital, Financial Capital and Livelihood. A composite index was developed using the mini-max normalization method. The index allowed the analysis of the vulnerability on a household, village and district level. 

From the analysis, we found that Chitwan district had a stable vulnerability across all waves of the survey (2006, 2009, 2012) with a mean HVI of 0.61. Kaski district exhibited mild variability, with a decrease in mean HVI from 2006 to 2009 and an increase back to the original level in 2012.
Mustang district showed relatively higher household vulnerability in 2006, a slight decrease in 2009, and stability in 2012. 

After the household vulnerability analysis, we investigated its relationship with Environmental Dependence. The household vulnerability index, calculated based on established equations, served as the dependent variable. Environmental dependence as a major determinant, consistently exhibited a significantly positive association with vulnerability. If the dependence persists further, it might have an adverse effect on the households as well as the environment. Household vulnerability increases because they are prone to environmental risk. From the environmental perspective, this dependence is a threat to sustainable environment as it will degrade the environment.

Additionally, the study attempted to study the association of Debt on vulnerability. Debt, as a coping mechanism, displayed a negative relationship with vulnerability, though statistically insignificant. Debt plays important role of reducing vulnerability of the households in multi-faceted manner. It acts as a safety nets during the times of shocks and crisis by providing immediate financial support. It also enables households to invest in productive activities which reduces vulnerability and increases resilience capacity of the households in the long-run. Further, it helps to strengthen the capitals of the households that make the household resilient. Capitals such as Financial and Social Capital.  

A high dependency ratio is characterized by a large proportion of dependents relative to the working-age population. A high dependency ratio has a positive effect on household vulnerability. Having a high dependency ratio leads to heightened vulnerability to economic shocks and poverty, perpetual cycles of financial instability and deprivation. Having higher number of elderly and children means that the adult working age family member needs to refrain from participating in the labor force in order to do the care work for the dependents. This in turn exacerbates the vulnerability of the households. 

Households in rural Nepal face a lot of shocks that impact their livelihoods severely. Natural disasters, economic downturns, or health shocks or any other idiosyncratic or co-variate shocks increase the vulnerability of the household. Moreover, shocks exacerbate pre-existing vulnerabilities, disproportionately affecting poor and vulnerable. Shocks have a multifaceted influence on the household and vulnerability. The result of the study confirms that households facing more number of shocks are more vulnerable. 

\subsection{Recommendations and Possible Extensions}

This paper builds national and sub-national level vulnerability assessment \cite{antwi2013characterising, aksha2019analysis, shahiestimating} by developing and applying a household vulnerability index to characterize the nature and inter-temporal dynamics of the household vulnerability across the distinct geographic regions of Nepal. The study identifies differential vulnerability across communities and households due to differences in capitals. The mountainous village of Mustang district has a higher vulnerability in rural areas, highlighting the importance of considering these differences when developing interventions to address specific vulnerabilities.

The study also highlights the positive relationship between environmental dependency and household vulnerability. Policies aimed at reducing dependency on the environment should be designed and implemented in rural settings. Community-level resilience programs, educational initiatives, and cross-sectoral collaboration could help foster sustainable development, resilience, and improved livelihoods in rural Nepalese communities. To reduce vulnerability, households should be encouraged to invest funds from debt in productive assets, education, healthcare, and social capital. Responsible borrowing and effective debt management strategies should be practiced to prevent excessive indebtedness and financial instability. Policies should enhance access to credit and financial services for vulnerable households to facilitate economic empowerment and resilience-building initiatives. Social safety nets and care facilities should be provided to alleviate financial strain on households with higher dependency ratios. Government investments in disaster preparedness, risk reduction, and resilient infrastructure are also needed to ensure quick relief and recovery.

The study's findings can be expanded by discussing the weighting of household vulnerability components with stakeholders. Future studies could involve stakeholder consultation on the weights of household vulnerability components.
