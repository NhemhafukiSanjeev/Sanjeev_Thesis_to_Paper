\section{Conclusion} \label{isect5}

In this paper, we examined the trajectory of the gender wage gap in rural and urban areas of Nepal by quantile-wise decomposing the gender wage gap into composition effects and structural effects. We found that the wage gap is converging for higher percentile groups while it is persisting or widening among lower earners, i.e., ``sticky floor'' phenomenon. The structural effect mirrors the slope of the total effect, whereas the composition effect amplifies the distribution uniformly across all percentiles in both urban and rural Nepal. In addition, we observed a notable trend of improvement in composition effect throughout the time period, with education progressing beyond gender parity. However, this improvement is overshadowed by the exacerbation of the structural effect, which persists even after adjusting for selection. Almost all of the wage gap is attributable to the differential returns to observed factors or to the unobserved factors, which in the worst case can be the situation of being a woman in an unfriendly social context. This situation differs from 1998, when the composition effect used to explain a considerable portion of the gap.\par

We investigated this divergence by looking at the household dynamics that affect female labour force participation. We found that improvement of a woman's education does not necessarily guarantee female labour market participation. Women's success is linked with spousal education level: a higher spousal education gap pushes women away from the job market, more so as they rise in the family hierarchy. Along with this gendered sorting in the labour market, there is a substantial gender discrepancy in time spent on household chores. Whether they are employed or not, women always contribute substantially more to home production, a trend that has hardly changed in the two decades we studied. Whereas in the same duration, men have contributed a very small time in home production consistently and there is almost zero substitution effect. This dual burden costs women the flexibility to participate in the job market. Addressing these structural issues requires more than simply providing women with higher education and improving their job skills.\par  

An avenue for future research involves incorporating the psychological attributes of workers and examining the impact and consequences of policy changes, such as affirmative action policies implemented by the government. Investigating how these policies influence both wage disparities and economic outcomes could offer valuable insights into the effectiveness of such initiatives and their implications for gender equity.\par






 
