\clearpage
\section{Results and discussion} \label{isect4}

\subsection{Empirical Results}
The study examines household vulnerability through regression estimates, focusing on environmental dependence, debt, dependency ratio, and shock. The results show that environmental income can be a liability for vulnerable households, while debt can reduce vulnerability. Debt and Dependency ratio is a liability factor for resilience. The study also includes shock as a control, finding that household vulnerability increases with the number of shocks experienced. Fixed effects controls have also been introduced in the regression models to account for the variation caused by the time-invariant variables such as district and VDC. The analysis employs three methods of panel data: Pooled OLS, Random Effects, and Fixed Effects. The results show a significant positive association between household vulnerability and environmental dependence.

Table \ref{tab:paneldata} presents the results of the Panel Data regression employed in the analysis. The first model is a simple Pooled OLS model where the dependent variable household vulnerability has been regressed by Environmental dependence. The result on the regression is that increase in Environmental dependence increases the household vulnerability. This simple regression has been the landmark of this analysis. Following the result, we extend the regression in the following models for Pooled OLS by adding control variables Debt, Dependency ratio and Shock. Further, we add Fixed effects of Year, District and VDC.

\begin{table}[htb] 
	\begin{center}
		\caption{Panel Data Regression} 
		\renewcommand{\arraystretch}{1.05}
		\resizebox{0.7\textwidth}{!}{%
			\begin{tabular}{@{\extracolsep{1pt}}lD{.}{.}{-3} D{.}{.}{-3} D{.}{.}{-3} D{.}{.}{-3} } 
				\\[-2ex]\hline 
				\hline \\[-2.5ex] 
				& \multicolumn{4}{c}{\textit{Dependent variable: Household Vulnerability}} \\ 
				\cline{2-5} \\
				[-2.6ex] & \multicolumn{1}{c}{POLS (1)} & \multicolumn{1}{c}{POLS (2)} & \multicolumn{1}{c}{RE} & \multicolumn{1}{c}{FE}\\ 
				\hline \\[-2.8ex] 
				Env. Dependence & 0.068^{***} & 0.045^{***} & 0.024^{**} & -0.004 \\ 
				& (0.012) & (0.012) & (0.011) & (0.013) \\ 
				& & & & \\ [-1.8ex]
				Debt &  & -0.0005 & -0.0004 & -0.0003 \\ 
				&  & (0.0003) & (0.0003) & (0.0003) \\ 
				& & & & \\ [-1.8ex] 
				Dependency ratio &  & 0.020^{***} & 0.018^{***} & 0.014^{***} \\ 
				&  & (0.002) & (0.002) & (0.002) \\ 
				& & & & \\ [-1.8ex] 
				Shock &  & 0.001 & 0.001 & 0.001 \\ 
				&  & (0.001) & (0.001) & (0.001) \\ 
				& & & & \\ [-1.8ex] 
				Constant & 0.557^{***} & 0.562^{***} & 0.583^{***} &  -  \\ 
				& (0.012) & (0.012) & (0.011) & -  \\ 
				& & & & \\[-2.5ex] 
				\hline \\[-3ex] 
				\textit{Fixed Effects} & & & & \\ [-1ex]
				Year & \multicolumn{1}{c}{No} & \multicolumn{1}{c}{Yes} & \multicolumn{1}{c}{Yes} & \multicolumn{1}{c}{Yes} \\ [-0.7ex]
				District & \multicolumn{1}{c}{No} & \multicolumn{1}{c}{Yes} & \multicolumn{1}{c}{Yes} & \multicolumn{1}{c}{Yes} \\ [-0.7ex]
				VDC & \multicolumn{1}{c}{No} & \multicolumn{1}{c}{Yes} & \multicolumn{1}{c}{Yes} & \multicolumn{1}{c}{Yes} \\ 
				\hline \\[-3ex] 
				\textit{Fit statistics} & & & & \\ [-1ex]
				Observations & \multicolumn{1}{c}{1,284} & \multicolumn{1}{c}{1,284} & \multicolumn{1}{c}{1,284} & \multicolumn{1}{c}{1,284} \\ [-0.7ex]
				R$^{2}$ & \multicolumn{1}{c}{0.024} & \multicolumn{1}{c}{0.131} & \multicolumn{1}{c}{0.095} & \multicolumn{1}{c}{0.042} \\[-0.7ex] 
				Adjusted R$^{2}$ & \multicolumn{1}{c}{0.023} & \multicolumn{1}{c}{0.125} & \multicolumn{1}{c}{0.089} & \multicolumn{1}{c}{-0.446} \\ 
				\hline 
				\hline \\ [-2.8ex] 	
			\end{tabular}
		}
		\parbox{\linewidth}{\scriptsize\textit{ \ \ \ \ \ \ \ \ \ \ \ \ \ \ \ \ \ \ \ \ \ Note: Standard errors in the parenthesis:} \ \ \ \ \ \ \ \ \ \ \ \ \ \ \ \ \ \ \ \ {$^{*}$p$<$0.1; $^{**}$p$<$0.05; $^{***}$p$<$0.01}} \\ \vspace{-0.2cm}
		\parbox{\linewidth}{\scriptsize\textit{\ \ \ \ \ \ \ \ \ \ \ \ \ \ \ \ \ \ \ \ \ Env. = Environmental}}
		\label{tab:paneldata} 
	\end{center}
\end{table}   


After running the Pooled OLS regression we ran the Lagrange Multiplier Test -  \cite{honda1988size} to test if there are time effects in our model. The Null hypothesis is that there are no time effects in the model. The result indicate that there are significant time effects in the model. The test statistic is 10.822 and the p-value is extremely small (0.000) which suggests that there is a strong evidence to reject null hypothesis. The result of the test is in the Appendix B1.\par  

Similarly, we also tested for the individual effects in the Pooled OLS model. We conducted F-test and found that there are individual effects (fixed effects) in the model. The results suggests strong evidence to reject the null hypothesis. The F-statistics is 2.4874 with degree of freedom df1=472 and df2=852. The p-value is extremely small (0.000). This implies that there are significant individual effects in the model. The results of the tests are in Appendix B2.\par 

Referring to the results of the test, we included the time factor (Year) as well as time-invariant factors District and VDC. Since, the result suggested that there is strong evidence that there are time as well as individual effects. So, we controlled the Year, District and VDC fixed effects.\par 

After obtaining the results from the Pooled OLS regression, we tested for the pool-ability to confirm if the cross-sectional unit in the panel has the same intercept or a different intercept. Also, whether or not it had different slopes. For this purpose, we employed Breusch and Pagan Lagrangian multiplier test \citep{breusch1980lagrange} to test the poolability of the data. It was confirmed that the panel data was not pool-able. So, Pooled OLS is not appropriate for the model. The result of the Breusch and Pagan Lagrange Multiplier test is in Appendix B3.\par 

Since the BP-LM test result indicate that p-value = 0.000, we conclude that the Pooled OLS model is not an efficient estimator for our data. So, we run Random Effects (RE) model. The results from the Random Effects (RE) model closely resemble those from the Pooled Ordinary Least Squares (POLS) model. The step-wise regression results for the Random effects (RE) is on the Table \ref{tab:randomeffect}. The model progression is similar to Pooled OLS model.

We also run the fixed effects (FE) regression. The result of the full specification FE model is in the Table \ref{tab:fixedeffect}. The model progression is similar to Pooled OLS and RE model. The first model is simple Random effects model       

After the FE regression, we conduct Hausman Specification test \citep{hausman1978specification}. The test assesses whether the coefficients estimated by the two models are significantly different. The test for our model suggest that fixed effect model is preferred over random effect model. The corresponding chi-square value is 25.48 which is relatively large with 6 degrees of freedom, and the adjoint probability was much less than 0.05. The result obtained for Hausman Specification test is in the Appendix B4.\par 

\subsubsection{Environmental Dependence and Vulnerability (OLS)}

\begin{table}[ht] 
	\renewcommand{\arraystretch}{1.1}
	\caption{Pooled OLS Regression}
	\resizebox{1.05\textwidth}{!}{%
		\begin{tabular}{@{\extracolsep{0.01pt}}lD{.}{.}{-3} D{.}{.}{-3} D{.}{.}{-3} D{.}{.}{-3} D{.}{.}{-3} D{.}{.}{-3} D{.}{.}{-3} } 
			\\[-1.6ex]\hline 
			\hline \\[-1.8ex] 
			& \multicolumn{7}{c}{\textit{Dependent variable:Household Vulnerability}} \\ 
			\cline{2-8} 		\\[-2.9ex] & \multicolumn{1}{c}{(1)} & \multicolumn{1}{c}{(2)} & \multicolumn{1}{c}{(3)} & \multicolumn{1}{c}{(4)} & \multicolumn{1}{c}{(5)} & \multicolumn{1}{c}{(6)} & \multicolumn{1}{c}{(7)}\\ 
			\hline \\[-1.8ex] 
			Env Dependence & 0.068^{***} & 0.068^{***} & 0.064^{***} & 0.062^{***} & 0.062^{***} & 0.049^{***} & 0.045^{***} \\ 
			& (0.012) & (0.012) & (0.012) & (0.012) & (0.012) & (0.012) & (0.012) \\ 
			& & & & & & & \\ 
			Debt &  & -0.001^{*} & -0.0004 & -0.0004 & -0.0004 & -0.0004 & -0.0005 \\ 
			&  & (0.0003) & (0.0003) & (0.0003) & (0.0003) & (0.0003) & (0.0003) \\ 
			& & & & & & & \\ 
			Depndency ratio &  &  & 0.022^{***} & 0.021^{***} & 0.021^{***} & 0.021^{***} & 0.020^{***} \\ 
			&  &  & (0.002) & (0.002) & (0.002) & (0.002) & (0.002) \\ 
			& & & & & & & \\ 
			Shock &  &  &  & 0.002^{*} & 0.002 & 0.001 & 0.001 \\ 
			&  &  &  & (0.001) & (0.001) & (0.001) & (0.001) \\ 
			& & & & & & & \\ 
			Constant & 0.557^{***} & 0.561^{***} & 0.550^{***} & 0.550^{***} & 0.552^{***} & 0.558^{***} & 0.562^{***} \\ 
			& (0.012) & (0.012) & (0.011) & (0.011) & (0.012) & (0.012) & (0.012) \\ 
			& & & & & & & \\ [-3.1ex]
			\hline \\[-2.5ex] 
			\textit{Fixed effects} & & & & & & & \\ 
			Year & \multicolumn{1}{c}{No} & \multicolumn{1}{c}{No} & \multicolumn{1}{c}{No} & \multicolumn{1}{c}{No} & \multicolumn{1}{c}{Yes} & \multicolumn{1}{c}{Yes} & \multicolumn{1}{c}{Yes} \\ 
			District & \multicolumn{1}{c}{No} & \multicolumn{1}{c}{No} & \multicolumn{1}{c}{No} & \multicolumn{1}{c}{No} & \multicolumn{1}{c}{No} & \multicolumn{1}{c}{Yes} & \multicolumn{1}{c}{Yes} \\ 
			VDC & \multicolumn{1}{c}{No} & \multicolumn{1}{c}{No} & \multicolumn{1}{c}{No} & \multicolumn{1}{c}{No} & \multicolumn{1}{c}{No} & \multicolumn{1}{c}{No} & \multicolumn{1}{c}{Yes} \\ 
			\hline \\[-2.5ex] 
			\textit{Fit statistics} & & & & & & & \\
			Observations & \multicolumn{1}{c}{1,284} & \multicolumn{1}{c}{1,284} & \multicolumn{1}{c}{1,284} & \multicolumn{1}{c}{1,284} & \multicolumn{1}{c}{1,284} & \multicolumn{1}{c}{1,284} & \multicolumn{1}{c}{1,284} \\ 
			R$^{2}$ & \multicolumn{1}{c}{0.024} & \multicolumn{1}{c}{0.027} & \multicolumn{1}{c}{0.105} & \multicolumn{1}{c}{0.107} & \multicolumn{1}{c}{0.108} & \multicolumn{1}{c}{0.126} & \multicolumn{1}{c}{0.131} \\ 
			Adjusted R$^{2}$ & \multicolumn{1}{c}{0.023} & \multicolumn{1}{c}{0.025} & \multicolumn{1}{c}{0.103} & \multicolumn{1}{c}{0.105} & \multicolumn{1}{c}{0.104} & \multicolumn{1}{c}{0.121} & \multicolumn{1}{c}{0.125} \\ 
			\hline 
			\hline \\[-1.8ex]  
		\end{tabular}
	}
	\textit{Note: Standard errors in the parenthesis} \hspace{2.52cm}{$^{*}$p$<$0.1; $^{**}$p$<$0.05; $^{***}$p$<$0.01} \\
	\textit{Env. = Environmental}
		\label{tab:pooledols} 
\end{table} 

Table \ref{tab:pooledols} presents the results of the Pooled OLS Regression results. The Pooled OLS Regression results show a significant positive association between household vulnerability and environmentally dependent factors. As environmental dependence increases, vulnerability also increases. Debt has a negative effect on vulnerability, but the regression coefficient remains unchanged. Dependency ratio and shock have positive effects on vulnerability, while shock has a positive influence. The main independent variable, environmental dependence, remains the same as in model 1, but the coefficient declines with the addition of control variables. The shock variable has a positive effect on vulnerability.

\subsubsection{Environmental Dependence and Vulnerability (RE)}
Table \ref{tab:randomeffect} presents the Random Effects Regression results of our estimates. The Random Effects Regression results show a significant positive association between household vulnerability and environmentally dependent factors. Control variables such as debt, dependency, and shock were included in models 2, 3, and 4. Debt had a negative effect on vulnerability in model 2, but lost significance in model 3, and had a positive effect in model 4. In models 5, 6, and 7, debt had a negative but insignificant effect.

\begin{table}[htb] 
	\caption{Random Effects Regression} 
	\renewcommand{\arraystretch}{1.5}
	\resizebox{1.\textwidth}{!}{% 
		\begin{tabular}{@{\extracolsep{0.01pt}}lD{.}{.}{-3} D{.}{.}{-3} D{.}{.}{-3} D{.}{.}{-3} D{.}{.}{-3} D{.}{.}{-3} D{.}{.}{-3} } 
			\\[-1.8ex]\hline 
			\hline \\[-2.9ex] 
			& \multicolumn{7}{c}{\textit{Dependent variable:Household Vulnerability}} \\ 
			\cline{2-8} \\[-7ex] 
			& \\
			[-1.8ex] & \multicolumn{1}{c}{(1)} & \multicolumn{1}{c}{(2)} & \multicolumn{1}{c}{(3)} & \multicolumn{1}{c}{(4)} & \multicolumn{1}{c}{(5)} & \multicolumn{1}{c}{(6)} & \multicolumn{1}{c}{(7)}\\ 
			\hline \\[-3.9ex] 
			Env. Dependence & 0.035^{***} & 0.035^{***} & 0.037^{***} & 0.035^{***} & 0.034^{***} & 0.026^{**} & 0.024^{**} \\ [-1.5ex]
			& (0.011) & (0.011) & (0.011) & (0.011) & (0.011) & (0.011) & (0.011) \\ [-3.5ex]
			& & & & & & & \\ 
			Debt &  & -0.001^{*} & -0.0003 & -0.0004 & -0.0004 & -0.0004 & -0.0004 \\ [-1.5ex]
			&  & (0.0003) & (0.0003) & (0.0003) & (0.0003) & (0.0003) & (0.0003) \\ [-3.5ex]
			& & & & & & & \\ 
			Dependency ratio &  &  & 0.020^{***} & 0.019^{***} & 0.019^{***} & 0.019^{***} & 0.018^{***} \\[-1.5ex] 
			&  &  & (0.002) & (0.002) & (0.002) & (0.002) & (0.002) \\ [-3.5ex]
			& & & & & & & \\ 
			Shock &  &  &  & 0.002^{**} & 0.001 & 0.001 & 0.001 \\ [-1.5ex]
			&  &  &  & (0.001) & (0.001) & (0.001) & (0.001) \\ [-3.5ex]
			& & & & & & & \\ 
			Constant & 0.588^{***} & 0.592^{***} & 0.576^{***} & 0.576^{***} & 0.580^{***} & 0.581^{***} & 0.583^{***} \\ [-1.5ex] 
			& (0.011) & (0.011) & (0.011) & (0.011) & (0.011) & (0.011) & (0.011) \\ [-4.5ex]
			& & & & & & & \\ 
			\hline \\[-5ex] 
			\textit{Fixed effects} & & & & & & & \\  \\[-6ex]
			Year & \multicolumn{1}{c}{No} & \multicolumn{1}{c}{No} & \multicolumn{1}{c}{No} & \multicolumn{1}{c}{No} & \multicolumn{1}{c}{Yes} & \multicolumn{1}{c}{Yes} & \multicolumn{1}{c}{Yes} \\ [-1.5ex]
			District & \multicolumn{1}{c}{No} & \multicolumn{1}{c}{No} & \multicolumn{1}{c}{No} & \multicolumn{1}{c}{No} & \multicolumn{1}{c}{No} & \multicolumn{1}{c}{Yes} & \multicolumn{1}{c}{Yes} \\ [-1.5ex]
			VDC & \multicolumn{1}{c}{No} & \multicolumn{1}{c}{No} & \multicolumn{1}{c}{No} & \multicolumn{1}{c}{No} & \multicolumn{1}{c}{No} & \multicolumn{1}{c}{No} & \multicolumn{1}{c}{Yes} \\ 
			\hline \\[-5ex] 
			\textit{Fit statistics} & & & & & & & \\ [-1.5ex]
			Observations & \multicolumn{1}{c}{1,284} & \multicolumn{1}{c}{1,284} & \multicolumn{1}{c}{1,284} & \multicolumn{1}{c}{1,284} & \multicolumn{1}{c}{1,284} & \multicolumn{1}{c}{1,284} & \multicolumn{1}{c}{1,284} \\ [-1.5ex]
			R$^{2}$ & \multicolumn{1}{c}{0.007} & \multicolumn{1}{c}{0.010} & \multicolumn{1}{c}{0.072} & \multicolumn{1}{c}{0.075} & \multicolumn{1}{c}{0.076} & \multicolumn{1}{c}{0.091} & \multicolumn{1}{c}{0.095} \\ [-1.5ex]
			Adjusted R$^{2}$ & \multicolumn{1}{c}{0.007} & \multicolumn{1}{c}{0.008} & \multicolumn{1}{c}{0.070} & \multicolumn{1}{c}{0.072} & \multicolumn{1}{c}{0.072} & \multicolumn{1}{c}{0.085} & \multicolumn{1}{c}{0.089} \\ 
			\hline 
			\hline  
		\end{tabular} 
	}
	\textit{Note: Standard errors in the parenthesis} \hspace{2.52cm}{$^{*}$p$<$0.1; $^{**}$p$<$0.05; $^{***}$p$<$0.01} 
	\textit{Env. = Environmental}
		\label{tab:randomeffect}
\end{table} 

Importantly, our major variable of interest Environmental dependence continued to have a positive effect across all the models.  Up to model 5, Environmental dependence displayed to have a positive effect on household vulnerability at 1\% significance level. However, the significance level reduced to 5\% in the following models. The loss in the significance is the effect of introduction of the time invariant fixed effects: District and VDC.

\subsubsection{Environmental Dependence and Vulnerability (FE)}
Table \ref{tab:fixedeffect} presents the regression estimates of Fixed effects model. Result from Model 1 suggest that there is a negative association between the household vulnerability and environmentally dependence. However, there is no significance of the effect. Here, no control variables have been included.  

\begin{table}[htb] 
	\caption{Fixed Effects Regression}  
	\renewcommand{\arraystretch}{1.5}
	\resizebox{1.1\textwidth}{!}{%
		\begin{tabular}{@{\extracolsep{5pt}}lD{.}{.}{-3} D{.}{.}{-3} D{.}{.}{-3} D{.}{.}{-3} D{.}{.}{-3} D{.}{.}{-3} D{.}{.}{-3} } 
			\\[-1.8ex]\hline 
			\hline \\[-3ex] 
			& \multicolumn{7}{c}{\textit{Dependent variable: Household Vulnerability}} \\ 
			\cline{2-8} 
			\\
			\\[-7.8ex] & \multicolumn{1}{c}{(1)} & \multicolumn{1}{c}{(2)} & \multicolumn{1}{c}{(3)} & \multicolumn{1}{c}{(4)} & \multicolumn{1}{c}{(5)} & \multicolumn{1}{c}{(6)} & \multicolumn{1}{c}{(7)}\\ 
			\hline \\[-3.98ex] 
			Env. Dependence & -0.003 & -0.003 & -0.001 & -0.001 & -0.004 & -0.004 & -0.004 \\ [-1.5ex]
			& (0.013) & (0.013) & (0.013) & (0.013) & (0.013) & (0.013) & (0.013) \\ [-3.5ex]
			& & & & & & & \\ 
			Debt &  & -0.0004 & -0.0003 & -0.0003 & -0.0003 & -0.0003 & -0.0003 \\ [-1.5ex]
			&  & (0.0004) & (0.0004) & (0.0004) & (0.0004) & (0.0004) & (0.0003) \\[-3.5ex] 
			& & & & & & & \\ 
			Dependency ratio &  &  & 0.015^{***} & 0.015^{***} & 0.014^{***} & 0.014^{***} & 0.014^{***} \\ [-1.5ex]
			&  &  & (0.003) & (0.003) & (0.003) & (0.003) & (0.002) \\ [-3.5ex]
			& & & & & & & \\ 
			Shock &  &  &  & 0.002^{*} & 0.001 & 0.001 & 0.001 \\ [-1.5ex]
			&  &  &  & (0.001) & (0.001) & (0.001) & (0.001) \\ [-4ex]
			& & & & & & & \\ \hline \\[-5ex] 
			\textit{Fixed Effexts} 	&  &  &  &  &  &  & \\ [-1.5ex]
			Year & \multicolumn{1}{c}{No} & \multicolumn{1}{c}{No} & \multicolumn{1}{c}{No} & \multicolumn{1}{c}{No} & \multicolumn{1}{c}{Yes} & \multicolumn{1}{c}{Yes} & \multicolumn{1}{c}{Yes} \\ [-1.5ex] 
			District & \multicolumn{1}{c}{No} & \multicolumn{1}{c}{No} & \multicolumn{1}{c}{No} & \multicolumn{1}{c}{No} & \multicolumn{1}{c}{No} & \multicolumn{1}{c}{Yes} & \multicolumn{1}{c}{Yes} \\ [-1.5ex]
			VDC & \multicolumn{1}{c}{No} & \multicolumn{1}{c}{No} & \multicolumn{1}{c}{No} & \multicolumn{1}{c}{No} & \multicolumn{1}{c}{No} & \multicolumn{1}{c}{No} & \multicolumn{1}{c}{Yes} \\ [-1.ex]
			\hline \\[-5ex] 
			\textit{Fit statistics} 	&  &  &  &  &  &  & \\ [-1.5ex]
			Observations & \multicolumn{1}{c}{1,284} & \multicolumn{1}{c}{1,284} & \multicolumn{1}{c}{1,284} & \multicolumn{1}{c}{1,284} & \multicolumn{1}{c}{1,284} & \multicolumn{1}{c}{1,284} & \multicolumn{1}{c}{1,284} \\ [-1.5ex]
			R$^{2}$ & \multicolumn{1}{c}{0.00005} & \multicolumn{1}{c}{0.002} & \multicolumn{1}{c}{0.034} & \multicolumn{1}{c}{0.037} & \multicolumn{1}{c}{0.042} & \multicolumn{1}{c}{0.042} & \multicolumn{1}{c}{0.042} \\ [-1.5ex]
			Adjusted R$^{2}$ & \multicolumn{1}{c}{-0.501} & \multicolumn{1}{c}{-0.500} & \multicolumn{1}{c}{-0.453} & \multicolumn{1}{c}{-0.450} & \multicolumn{1}{c}{-0.446} & \multicolumn{1}{c}{-0.446} & \multicolumn{1}{c}{-0.446} \\ [-0.5ex]
			\hline 
			\hline \\[-2.8ex] 
		\end{tabular} 
	}
	\textit{Note: Standard errors in the parenthesis} \hspace{2.52cm}{$^{*}$p$<$0.1; $^{**}$p$<$0.05; $^{***}$p$<$0.01} \\
	\textit{Env. = Environmental}
	\label{tab:fixedeffect}
\end{table}

Environmental dependence has a negative and insignificant effect up to model 6, but a positive and significant effect in model 7. Results found that debt has a negative but insignificant impact on vulnerability, while dependency ratio has a positive effect. Shock has a positive and significant effect.

\subsection{Discussions}
\subsubsection{Household Vulnerability Index}

We assessed household vulnerability in three distinct physio-graphical regions of Nepal: Mountain, Mid-hill, and Lowland. The household vulnerability index (HVI) was constructed for the Kunjo and Lete VDC of Mustang, Hemja VDC of Kaski, and Chainpur VDC of Chitwan. The HVI ranged from 0.61 to 0.65 across the survey districts in 2006. The highest vulnerability was observed in the Kunjo and Lete VDC of Mustang, which persisted across all waves of the survey years. This suggests that the mountainous region remains the most vulnerable, as it is often characterized by geographical and infrastructural difficulties, scarce resources, and socio-economic disparities.

On the other hand, the Hill and Lowland regions enjoy more resources and infrastructural advantages, making the HVI less than that of the mountains. Hemja VDC of Kaski maintained a similar level of vulnerability with a 0.62 HVI. By 2012, some VDCs experienced lowered vulnerability, while Chainpur VDC of Chitwan experienced a slight decrease in vulnerability. The study presented the District and VDC level HVI in radar charts figure \ref{fig:distictlevelhvi} \& \ref{fig:vdchvisummary} and presented the component of the household vulnerability index in a stacked line with markers in figures (\ref{fig:hvichainpurComponent} - \ref{fig:hviletecomponents}) for each VDC across all survey years. We also present the persistence and transience of the vulnerability positions of each VDCs using a Sankey diagram in figure \ref{fig:sankey}. 

\subsubsection{Household Vulnerability and Environmental Dependence}
The study examines the relationship between household vulnerability and environmental dependence in VDCs and districts across survey years. It employs Panel data regression techniques, including Pooled OLS, Random Effects (RE), and Fixed Effects (FE) regression techniques. The dependent variable is household vulnerability, while the independent variables include environmental dependence, debt, dependency ratio, and shock. Time-invariant fixed effects were added in model 5-7 to change the coefficients of the variables.

The results confirm the findings of previous studies \citep{angelsen2015environmental, abbas2018sustainable, gentle2014differential} which show that environmental resource dependence has a positive effect on household vulnerability, especially in the context of changing climate and climatic hazards. The increasing dependence on environment for livelihoods can exacerbate vulnerability by increasing sensitivity to climate change, limited diversification of livelihood sources, increased resource degradation, and exposure to extreme idiosyncratic and co-variate shocks.

Debt is a crucial coping mechanism for households, particularly when faced with shocks \citep{rabbani2021role}. It provides financial resources to cope with shocks and invest in resilience-building measures. Debt not only acts as a cushion against shocks but also as a source of investment for productive activities, expenditure in education and healthcare, and building physical, financial, and social capital. The empirical analysis suggests that debt has a negative effect on household vulnerability, suggesting that it helps reduce vulnerability.

The study reveals that household vulnerability is significantly influenced by the dependency ratio, which is characterized by financial strain, reduced labor productivity, limited social support, and inter-generational transmission of poverty \citep{rabbani2021role, sun2020nexus}. Idiosyncratic shocks, such as disruptions in livelihoods, depletion of assets, food insecurity, malnutrition, health impacts, healthcare costs, and psychological trauma, also increase vulnerability \cite{buhler2018shocks, barua2020impact, volker2010rural}. The study finds that environmental dependence positively influences household vulnerability, while dependency ratio and shock variables also have a positive effect. Debt, however, has a negative effect on household vulnerability.
