\begin{abstract}
This paper studies the evolution of the gender wage gap and looks at its source arising from the household-level dynamics. First, we decompose the selection-adjust-ed gender wage gap distribution over three rounds of Nepal Labor Force Surveys (1998-2018) and discuss disparities over time. Despite achieving parity in human capital, the gap stagnates for below-median earners but converges at higher wage tiers in urban and rural areas, showing a ``sticky floor'' nature. Moreover, by 2018, the source of the gap diverged - almost all of the gap was due to unobserved characteristics. Second, we test the implications of the household decision-making model on female labor force participation using the 2011 national census. We find that a higher spousal potential earning gap hinders women from being employed. Also, women allocate substantially more time to household chores, indifferent to the employment status, and effectively experience a ``double burden'' of work when employed. These results point out that improving human capital is an exhausted strategy. As long as women's participation is a derivative of men's earning potential and time allocations are skewed against women, the convergence of the gap remains challenging.\par
\vspace{.1in}
\noindent\textbf{Keywords:} Labor participation, Quantile-Copula, Distributional decomposition, Selection bias, Gendered Labor market outcomes, Education gap.\\
\noindent\textbf{JEL Codes:} J-No Idea
\bigskip
\end{abstract}