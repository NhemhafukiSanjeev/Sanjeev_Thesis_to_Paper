\section{Introduction}\label{isect1}
Understanding the outcomes that women face when they participate in the labour market is key to designing gender-equitable labour policies -- especially in the global south, where women still face a myriad of challenges when they want to join or have joined the labour market. Globally, there has been a convergence in female participation and wage rates~\citep{WEF2020, OECD2023}; however, the pace of convergence has been stagnating in advanced economies~\citep{Olivetti2016, Blau2017}. This stagnation begs the question of whether the `low-hanging fruits' to achieve this goal have largely been exhausted in developed economies, leaving only politically sensitive or economically costly gender-parity policy measures. If developed economies -- with their greater financial strength and better-quality institutions -- struggle to sustain progress, it is imperative for developing countries to book-keep the factors driving the wage gap and to critically examine which aspects should be addressed in order to avoid a similar fate.\par

Our contribution to this extensive literature is twofold. The first contribution of this paper is to decompose the selection-adjusted wage distribution over two decades in a developing country, namely Nepal. The second contribution is to understand the important channels of structural bias, the intra-household gender education gap and the differential time allocation in home production, disadvantaging (girls and) women.\par

Generally, either due to societal expectations or personal choice, women allocate a larger share of time in home production, which increases women’s reservation wage and makes them less likely to participate in the job market. As a result, women’s participation in the job market invariably suffers from the sample selection issue that was first identified and addressed by~\citet{Heckman1974} and~\citet{Gronau1974}. Consequently, women who participate in the job market may not represent the overall female working-age population, as only women with a certain set of characteristics may join the job market. Several approaches have been developed in the literature to address this selection bias~\citep{Buchinsky1998, Blau2006, Blundell2007, Mulligan2008, Huber2015, Bar2015, Maasoumi2017, Arellano2017}. We opt for the quantile-copula approach of~\citet{Arellano2017} that jointly estimates wage and participation equations after explicitly modeling the correlation between the unobservables of both equations. Methodologically, this is less restrictive than the available alternatives, and the use of quantile regression helps to extract the entire wage distribution. Subsequently, we employ~\citet{Chernozhukov2013} to decompose the net difference between male and female wage distributions into composition and structure effects: the former is the wage gap resulting from differences in the individual characteristics of men and women, and the latter is the difference due to the varying returns of those characteristics.\par

The three rounds of the Nepal Labour Force Surveys (1998-2018) cover some of the major events that shaped the current Nepalese labour market. The decade-long armed conflict, which started in 1996 and claimed 14,242 lives~\citep{Joshi2015}, hindered the economic growth of the country. As a result, during the conflict and the post-conflict transition, the national economy struggled to accommodate the growing youth population, fueling the rise of international labour migration~\citep{libois2016households}. The out-migration rate surged after the conflict; although peaking in 2013/2014 and declining slightly thereafter~\citep{MoLE/GoN2013, MFES/GoN2020}, it continues to provide a sizable remittance inflow (up to a quarter of gross domestic product). Immediately after the conflict, the interim constitution of 2007 introduced a reservation system in public institutions for women and marginalised groups in society~\citep{Subedi2022, Mainali2017}. At the same time, the economic structure transformed from subsistence agriculture to service-based, without developing a significant industrial base~\citep{Sapkota2013}. This change led to a proliferation of service-sector jobs -- primarily in education, health, and finance -- in the newly liberalised parts of the economy. Thus, over the two decades of data coverage by the survey, the first half incorporate armed conflict and minimal job market changes, while the second half entails peak out-migration, the rapid growth of the service sector, and the implementation of a reservation system.\par

We find notable trends of gender wage convergence among high-earners but a widening or stagnant gap among low-earners. A ``sticky floor'' rather than the ``glass ceiling'' seems to be a more apt characterization of this phenomenon. Female labour force participation declined from 30.2\% in the agriculture-led job market of 1998 to 16.9\% in 2018, when the service sector dominated available jobs. This change in the nature of available jobs, along with affirmative action policies mandating 33\% women participants, selectively benefited educated women at the upper end of the wage distribution. The advantage of education in the service sector led to a significant influx of educated women into the labour force over the course of two decades. As a result, the contribution of the composition effect to the wage gap across the entire wage distribution completely vanished by 2018. Overall, in these two decades, wage disparities remarkably shifted towards being mostly structurally driven, rather than compositional one.\par

The qualitative change in the nature of the gap, i.e., a very meager compositional gap but nearly total structural gap, implies that improving human capital alone will not shrink the gap. The culprit, structural effect, stems from two sources: first, differing returns to observed characteristics; second, unobserved characteristics in the wage equation. With regard to the second source, unobserved characteristics can be any variables that have been studied in the literature -- like personal preferences~\citep{Wiswall2017, LeBarbanchon2020}, household dynamics~\citep{Bertrand2015, Goldin2017}, job characteristics~\citep{Manning2008, Card2015}, and societal structures~\citep{Givord2015, Lippmann2020, Goldin2006, Becker2019}, among others -- that affect female job market participation and outcomes. To understand the increasing trend toward structure effects, especially in urban areas, we look at the household-level dynamics and examine how they prevent women from participating in the labour force. The stylized household decision-making model of~\citet{Cortes2020} provides some interesting predictions: specifically, if wives have or are presumed to have an advantage in household tasks, they are less likely to participate in the job market vis-\'{a}-vis their husbands; moreover, if husbands have a larger advantage in the job market, wives are even more directed toward household chores.\par

We test this prediction in the national census (2011) by looking into the relationship between differential earning potential and employment status. We find that the spousal education gap, our proxy for difference in earning potential, hinders female job market participation and promotes sorting women into home production, especially after marriage. The negative effect of the spousal education gap increasingly overshadows the gains from years of schooling, when single women first marry and later become the spouse of the head of a household. Interestingly, the spousal education gap does not obstruct female participation in own-account work -- only their employment in the labour market.\par

Furthermore, we find this observed gendered sorting to be consistent with the time allocation in home production. Women allocate a similar amount of time doing household chores in across the three datasets: a living standard survey and two labour force surveys covering 2008 to 2018. In this time frame, women’s time declined by 40 minutes from the previous 142 minutes, which when examined alongside the employment status looks hollow. Men, regardless of employment status, contribute very little, but women consistently work almost at the same level. The observed decline does not seem to originate from gender balancing in home production but might have come about through widespread adoption of home appliances and infrastructure development, like increased access to piped drinking water, which is beyond the scope of this paper. In combination, these results confirm that household dynamics are important part of increasing structural effect when considering selection.\par

The paper is organised as follows. In section 2, we review selection adjustment methods and describe the estimation strategy and datasets used in this paper. In section 3, we describe the changes in the Nepalese labor market between 1998 and 2018 along with changes in workforce characteristics. Section 4 presents the results and discussions of decomposition, selection, earning potential, and time use. In section 5, we conclude with possible extensions of this research.\par                                                                                                                                                                                                                                                        

