\section{Introduction}\label{isect1}
According to The \cite{world2018rural} the global rural population constituted 43.5\% 
of the global population. The prevailing trend indicates a decline in the rural population is declining attributed to social, economic, technological, infrastructure, and environmental influences as discussed by \citep{jaszczak2018phenomenon}. The \cite{UNDP} projects that, by 2050, sixty-eight percent of the world’s population will be urban by 2050.
While scholars and institutions project a decline in the share of the rural population, the number of people living in rural areas still remains significant. The global rural population is estimated to be 3.4 billion, according to \cite{Worldbank2022}. The population residing in rural areas faces considerable vulnerability, as highlighted by \cite{acharya2008dimension}.  In terms of poverty, a striking 80 percent of those living in extreme poverty are found in rural areas \citep{world2021state}. Additionally, the escalating risks of climate change disproportionately affect rural populations \citep{Researchoverview2022}, posing a significant threat to their livelihoods, especially given the heavy dependence of many rural households on the natural environment. This phenomenon poses a massive threat to the poor rural livelihoods \citep{pelser2022climate}. \cite{angelsen2014environmental} contends that the natural environment such as forests and other natural areas, are crucial for sustaining rural livelihoods.\par

The rural population is increasingly falling into poverty, and those left behind are becoming more challenging to reach \citep{UN2019}. This trend has contributed to a rise in migration from rural areas \citep{lazarte2017understanding}. Environmental reasons, along with political and economic, stands among the drivers of migration \citep{mcnamara2016insecure}. In some instances, migrants are unfairly blamed for urban poverty \citep{tacoli2015rural}. Consequently, this imbalance in rural-urban resource distribution results in the mismanagement of opportunities and resources in rural areas. Simultaneously, it fuels heightened competition for urban resources, leading to their scarcity \citep{artuso2011state}. \par

Against this backdrop, it becomes imperative to investigate rural areas, their inhabitants, and their means of sustaining themselves to develop a more comprehensive understanding of rural economies. The residents of these areas predominantly rely on their surrounding environment, with natural resources assuming a pivotal role in their livelihoods \citep{nawrotzki2012natural}. Frequently, income generated from nature serves as a crucial safety net during periods of deficiencies in other livelihood activities, supporting immediate consumption needs and offering a potential pathway out of poverty \citep{angelsen2003exploring}. However, relying on environment also may contribute to the vulnerability of the households. Therefore, a thorough examination of the sources of livelihood in rural regions and the extent to which households are reliant on the environment becomes crucial.\par

The most frequently employed frameworks for rural livelihood and vulnerability analysis are \cite{anani1999sustainable}, \cite{dfid1999sustainable}, and \cite{ellis1999rural}. These frameworks serve as foundational tools for scrutinizing rural households within specific contexts and their livelihood resources. The Sustainable Livelihood Approach, incorporating key elements such as livelihood resources, vulnerability contexts, institutional processes, livelihood strategies, and outcomes, provides a comprehensive structure for rural livelihood analysis \citep{walelign2017dynamics}. Numerous studies have utilized frameworks like the Sustainable Livelihood Framework, Livelihood Assets Framework, Vulnerability Framework, and others to conduct in-depth analyses.\par

\cite{nawrotzki2012natural} underscores the central importance of natural resources in rural livelihoods through a thorough analysis. Building on this, \cite{diaz2019livelihood} establishes that capital assets play a crucial role in determining the livelihood strategies adopted by small-scale farmers. In addressing poverty in rural areas, \cite{mukotami2014rural} advocates for the promotion of non-farm activities, emphasizing that the poorest rural groups face limited opportunities for diversification, hindering the accumulation of resources for investment purposes. Furthermore, \cite{cavendish2008poverty}, in exploring the link between environmental income and inequality, identifies access to non-environmental cash income as the most significant contributor to rural inequality.\par

\cite{charlery2015assessing} employs a decomposition method to distinguish between stochastic and structural poverty within households, utilizing both income data and assets index. The study reveals that those classified as income poor exhibit a higher dependence on environmental resources in rural Nepal. In a related vein, \cite{walelign2017dynamics} employs clustering to identify various remunerative strategy groups, highlighting a higher reliance on environmental resources within the cluster employing the least remunerative strategies. The author suggests that Nepalese rural households in an upward transition phase show a reduction in environmental dependency, emphasizing the importance of enhancing poverty reduction strategies.\par

In the study conducted by \cite{walelign2020environmental}, it is observed that rural Nepalese households with high reliance on environmental resources exhibit lower income and asset endowments. Conversely, households with lower environmental reliance fare better in terms of both income and assets. Adding to this perspective, \cite{walelign2021poverty} highlights the impact of poor infrastructures in mountainous areas of Nepal, resulting in households having fewer assets and lower income compared to their counterparts in mid-hills and lowlands. Furthermore, \cite{chhetri2022importance} concludes that forest and environmental income remain the primary source of income and livelihoods for poor and marginalized households in Nepal, with a notable decrease in forest and environmental incomes as household income increases.\par

A substantial body of scholarly literature, spanning both international and national contexts, has delved into the exploration of factors influencing the livelihood strategies adopted by rural households across diverse countries. For instance, \cite{angelsen2014environmental} scrutinizes the determinants of household income across 8,000 households in 24 developing nations. The econometric model employed in the analysis incorporates factors such as household characteristics, assets, shocks, institutions, location, and site-level economic factors. In a related study, \cite{emeru2022determinants} identifies determinants of livelihood diversification strategies, including the age of the household, education status, family size, access to credit, market access, and positive impacts from training and extension services. Similarly, \cite{amevenku2019determinants} underscores the significance of marital status of the household head, the duration of food shortages experienced per year, access to credit and extension services, distance to regular markets and district capitals, as well as experience in fishery, as major determinants influencing livelihood strategies.\par

In the specific context of rural Nepal, there is a noticeable scarcity of literature assessment of the household vulnerability and environmental dependence. Furthermore, a critical gap exists in the identification and analysis of households that are vulnerable which is a crucial step to help policymakers foresee prospective routes for different segments, and thus frame out interventions more effectively. To address these gaps, this study endeavors to provide a comprehensive analysis, aiming to contribute valuable insights and enhance our understanding household vulnerability and its relationship with the environmental dependence in the rural landscapes of Nepal.\par

The paper is organised as follows. In section 2, we review household vulnerability and environmental facets of it. We describe the estimation strategy and datasets used in this paper. In section 3, we describe the changes in the Nepalese labor market between 1998 and 2018 along with changes in workforce characteristics. Section 4 presents the results and discussions of decomposition, selection, earning potential, and time use. In section 5, we conclude with possible extensions of this research.\par                                                                                                                                                                                                                                                        

