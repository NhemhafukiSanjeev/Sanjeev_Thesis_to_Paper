\clearpage % Start content on a new page
\begin{center}
	\section*{\large{CHAPTER I \\ \vspace{-0.3cm} INTRODUCTION}}
\end{center}
\addcontentsline{toc}{section}{\textbf{CHAPTER I:  INTRODUCTION}}
\renewcommand{\thepage}{\arabic{page}} 
\setcounter{page}{1}
This chapter presents the study’s background, statement of the problem,
research question, and thesis objectives. In addition, it highlights the significance
and limitations of the study.\\
\subsection*{1.1 Background of the Study}
\addcontentsline{toc}{subsection}{1.1 Background of the Study}
\renewcommand{\thepage}{\arabic{page}} 
\ \ \ According to The World \cite{world2018rural} the global rural population constituted 43.5\% 
of the global population. The prevailing trend indicates a decline in the rural population is declining attributed to social, economic, technological, infrastructure, and environmental influences as discussed by \citep{jaszczak2018phenomenon}. The \cite{UNDP} projects that by 2050, sixty-eight percent of the world’s population will be urban by 2050.
While scholars and institutions project a decline in the share of the rural population, the number of people living in rural areas still remains significant.

The population residing in rural areas faces considerable vulnerability, as highlighted by \cite{acharya2008dimension}. The global rural population is estimated to be 3.4 billion, according to \cite{Worldbank2022}. In terms of poverty, a striking 80 percent of those living in extreme poverty are found in rural areas \citep{world2021state}. Additionally, the escalating risks of climate change disproportionately affect rural populations \citep{Researchoverview2022}, posing a significant threat to their livelihoods, especially given the heavy dependence of many rural households on the natural environment. This phenomenon poses a massive threat to the poor rural livelihoods \citep{pelser2022climate}. \cite{angelsen2014environmental} contends that the natural environment such as forests and other natural areas, are crucial for sustaining rural livelihoods. So environmental reasons, along with political and economic, stands amongst the drivers of migration \citep{mcnamara2016insecure}.

The rural population is increasingly falling into poverty, and those left behind are becoming more challenging to reach \citep{UN2019}. This trend has contributed to a rise in migration from rural areas \citep{lazarte2017understanding}. In some instances, migrants are unfairly blamed for urban poverty \citep{tacoli2015rural}. Consequently, this imbalance in rural-urban resource distribution results in the mismanagement of opportunities and resources in rural areas. Simultaneously, it fuels heightened competition for urban resources, leading to their scarcity \citep{artuso2011state}. 

Against this backdrop, it becomes imperative to investigate rural areas, their inhabitants, and their means of sustaining themselves to develop a more comprehensive understanding of rural economies. The residents of these areas predominantly rely on their surrounding environment, with natural resources assuming a pivotal role in their livelihoods \citep{nawrotzki2012natural}. Frequently, income generated from nature serves as a crucial safety net during periods of deficiencies in other livelihood activities, supporting immediate consumption needs and offering a potential pathway out of poverty \citep{angelsen2003exploring}. However, relying on environment also may contribute to the vulnerability of the households. Therefore, a thorough examination of the sources of livelihood in rural regions and the extent to which households are reliant on the environment becomes crucial.

The most frequently employed frameworks for rural livelihood and vulnerability analysis are \cite{anani1999sustainable}, \cite{dfid1999sustainable}, and \cite{ellis1999rural}. These frameworks serve as foundational tools for scrutinizing rural households within specific contexts and their livelihood resources. The Sustainable Livelihood Approach, incorporating key elements such as livelihood resources, vulnerability contexts, institutional processes, livelihood strategies, and outcomes, provides a comprehensive structure for rural livelihood analysis \citep{walelign2017dynamics}. Numerous studies have utilized frameworks like the Sustainable Livelihood Framework, Livelihood Assets Framework, Vulnerability Framework, and others to conduct in-depth analyses.

\cite{nawrotzki2012natural} underscores the central importance of natural resources in rural livelihoods through a thorough analysis. Building on this, \cite{diaz2019livelihood} establishes that capital assets play a crucial role in determining the livelihood strategies adopted by small-scale farmers. In addressing poverty in rural areas, \cite{mukotami2014rural} advocates for the promotion of non-farm activities, emphasizing that the poorest rural groups face limited opportunities for diversification, hindering the accumulation of resources for investment purposes. Furthermore, \cite{cavendish2008poverty}, in exploring the link between environmental income and inequality, identifies access to non-environmental cash income as the most significant contributor to rural inequality.

\cite{charlery2015assessing} employs a decomposition method to distinguish between stochastic and structural poverty within households, utilizing both income data and assets index. The study reveals that those classified as income poor exhibit a higher dependence on environmental resources in rural Nepal. In a related vein, \cite{walelign2017dynamics} employs clustering to identify various remunerative strategy groups, highlighting a higher reliance on environmental resources within the cluster employing the least remunerative strategies. The author suggests, based on this finding, that Nepalese rural households in an upward transition phase show a reduction in environmental dependency, emphasizing the importance of enhancing poverty reduction strategies. 

In the study conducted by \cite{walelign2020environmental}, it is observed that rural Nepalese households with high reliance on environmental resources exhibit lower income and asset endowments. Conversely, households with lower environmental reliance fare better in terms of both income and assets. Adding to this perspective, \cite{walelign2021poverty} highlights the impact of poor infrastructures in mountainous areas of Nepal, resulting in households having fewer assets and lower income compared to their counterparts in mid-hills and lowlands. Furthermore, \cite{chhetri2022importance} concludes that forest and environmental income remain the primary source of income and livelihoods for poor and marginalized households in Nepal, with a notable decrease in forest and environmental incomes as household income increases.

A substantial body of scholarly literature, spanning both international and national contexts, has delved into the exploration of factors influencing the livelihood strategies adopted by rural households across diverse countries. For instance, \cite{angelsen2014environmental} scrutinizes the determinants of household income across 8,000 households in 24 developing nations. The econometric model employed in the analysis incorporates factors such as household characteristics, assets, shocks, institutions, location, and site-level economic factors. In a related study, \cite{emeru2022determinants} identifies determinants of livelihood diversification strategies, including the age of the household, education status, family size, access to credit, market access, and positive impacts from training and extension services. Similarly, \cite{amevenku2019determinants} underscores the significance of marital status of the household head, the duration of food shortages experienced per year, access to credit and extension services, distance to regular markets and district capitals, as well as experience in fishery, as major determinants influencing livelihood strategies.

In the specific context of rural Nepal, there is a noticeable scarcity of literature assessment of the household vulnerability and environmental dependence. Furthermore, a critical gap exists in the identification and analysis of households that are vulnerable which is a crucial step to help policymakers foresee prospective routes for different segments, and thus frame out interventions more effectively. To address these gaps, this study endeavors to provide a comprehensive analysis, aiming to contribute valuable insights and enhance our understanding household vulnerability and its relationship with the environmental dependence in the rural landscapes of Nepal.

\subsection*{1.2 Statement of the Problem }
\addcontentsline{toc}{subsection}{1.2 Statement of the Problem }
\renewcommand{\thepage}{\arabic{page}}
Developing countries are often characterized having majority of rural population. Many communities in developing countries heavily rely on natural resources and the environment for their rural livelihood strategies \citep{adger2000social, ahmadpour2020factors, chambers1992sustainable, ellis1999rural}. However, past literature has not thoroughly studied environmental dependency and its influence on household vulnerability. This could hinder the provision of a more accurate representation of environmental dependency, potentially limiting our understanding of the true impact of environmental changes on rural household vulnerability. Consequently, this gap in the literature may impede the development of effective policies to mitigate these effects.

Moreover, although \citep{mao2020rural, shan2020determinants, lorato2019determinants} have conducted some analysis on the determinants of rural livelihood strategies, there is a compelling need for more in-depth examination to gain a comprehensive understanding of the relationship with household vulnerability of these strategies, particularly those related with environment and nature. This entails a thorough investigation into the social, economic, and environmental factors that shape the decisions and choices of rural households and communities. Such a nuanced analysis is crucial for developing a holistic perspective on the intricate dynamics that influence policy and decision making.

In summary, the absence of a comprehensive understanding of environmental dependency and the household vulnerability determinants presents a substantial challenge to sustainable development and poverty reduction efforts in numerous rural areas in Nepal. It is essential to study the environmental dependency, particularly in the context of rapidly changing climatic environment and its influence on livelihoods and vulnerability. Further, there is a very limited studies that have assessed the transition of households to new state from old state of vulnerability. The panel structure of the dateset allows this study to assess the time-varying vulnerability transition from one state to another and assess how have the factors played role in the transience and persistence. It is imperative to address these knowledge gaps to formulate effective policies and interventions that can support rural communities and enhance their resilience in response to environmental changes. This proactive approach is essential for fostering sustainable development and improving the well-being of rural populations.

\subsection*{1.3 Research Questions}
\addcontentsline{toc}{subsection}{1.3 Research Questions }
\renewcommand{\thepage}{\arabic{page}}
The research question centers around the necessity for a more comprehensive understanding of the vulnerability of impoverished rural households and the factors influencing it. Notably, there is a gap in the assessment of household vulnerability based on the various components that contribute to the household's vulnerability, restricting a nuanced comprehension of the actual impact of the factors. Further, there is a gap of studies which studies the vulnerability positions of the households relative to the earlier positions. Moreover, a deeper analysis is warranted to scrutinize the social, economic, and environmental factors shaping the decisions of rural households and communities. Addressing these knowledge gaps is pivotal for developing policies and interventions that foster the resilience of rural communities in the face of environmental change, thereby supporting sustainable development and poverty reduction efforts in rural areas.
\vspace{1cm}
\newline
\textbf{Research Questions:} 
\begin{enumerate}
	\item[(i)] \parbox[t]{\linewidth}{What are the differences in household vulnerability to shocks and crisis across households in rural Nepal?}
	\item [(ii)] \parbox[t]{\linewidth}{What are the factors that play a role in determining the vulnerability to shocks and crisis of rural households in Nepal?}
\end{enumerate}
\subsection*{1.4 Objectives of the Study }
\addcontentsline{toc}{subsection}{1.4 Objectives of the Study   }
\renewcommand{\thepage}{\arabic{page}}
\textbf{General Objectives}
\\
The study aims to assess the household vulnerability to shocks and crisis and its determinants of the households in rural Nepal. To achieve the objective, two broad objectives has been set:
\begin{enumerate}
	\item[(i)] \parbox[t]{\linewidth}{Provide an overview of rural household vulnerability across households.}
	\item [(ii)] \parbox[t]{\linewidth}{Determine the factors that affect the household vulnerability with a focus on rural household's reliance on environmental resources.}
\end{enumerate}
\vspace{1cm}

\textbf{Specific Objectives} 
\\
To achieve the broad objective, two specific objective has been set:
\begin{enumerate}
	\item[(i)] \parbox[t]{\linewidth}{Compare the extent of rural household vulnerability to shocks and crisis within and across the households in selected villages from rural Nepal. Analyze the differences in the degree of vulnerability among these strategies.}
	
	\item[(ii)] \parbox[t]{\linewidth}{Identify and analyze the primary determinants influencing the rural household vulnerability to shocks and crisis in Nepal, with a particular emphasis on the environmental reliance.} 
\end{enumerate}  \\ \vspace{0.5cm}

\subsection*{1.5 Significance of the Study}
\addcontentsline{toc}{subsection}{1.5 Significance of the Study}
\renewcommand{\thepage}{\arabic{page}}
In the realm of environmental dependency across household strategy categories, the agricultural environment-based strategy group demonstrates the highest levels for both poverty incidence and environmental dependency, as noted by \cite{walelign2016livelihood}. Despite this, there is a notable dearth of research on how this dependence could be a factor that contribute to vulnerability in rural communities within the context of Nepal. Furthermore, no study has specifically addressed the relationship between the household vulnerability and environmental dependence.
The present study capitalizes on the panel characteristics of the data set. By utilizing established econometric tools, this thesis aims to fulfill its objectives. The findings of this study are anticipated to provide insights into the nature and differences of household vulnerability and environmental reliance in Nepal's rural communities. This information holds significance for policymakers as they assess which policies to prioritize in efforts to enhance the well-being of rural communities.

\subsection*{1.6 Scope and Limitations of the Study}
\addcontentsline{toc}{subsection}{1.6 Scope and Limitations of the Study}
\renewcommand{\thepage}{\arabic{page}}
\textbf{Scope:}\\
This study aims to examine the determinants of rural household vulnerability based on various assets of the hosueholds in the Chitwan (lowlands), Kaski (mid-hills), and Mustang (mountains)
of Nepal. The study will use a 3-wave panel data set collected in 2006, 2009, and 
2012, and will analyze the relationships between household demographics, assets, 
income, and their impact on rural household vulnerability. \\

The study investigates the array of assets available to households within the research area, encompassing financial, physical, social, and human resources. It delves into the variability of households' vulnerability compared to others, considering factors such as income, education, access to services, and social networks. Key determinants of household vulnerability are identified, including economic status, environmental conditions, health, and social capital. Furthermore, the study examines how these determinants fluctuate across diverse regions and evolve over time, providing insights into the dynamic nature of vulnerability within different contexts.
This study aims at assessing the household vulnerability and environmental dependence of the rural households in Nepal.\\
\newline
\textbf{Limitations:}
\begin{enumerate}
	\item[(i)] \parbox[t]{\linewidth}{The study is limited to three districts in Nepal and may not be representative of 
		other regions or countries.}
	\item[(ii)] \parbox[t]{\linewidth}{The analysis will rely on secondary data.}
	\item[(iii)] \parbox[t]{\linewidth}{The study will focus on a limited set of variables and may  and may leave out consideration of other equally important factors that affect the household vulnerability.}\\
\end{enumerate}

\subsection*{1.7 Organizations of the Study}
\addcontentsline{toc}{subsection}{1.7 Organizations of the Study}
\renewcommand{\thepage}{\arabic{page}}
The following chapter provides an overview of the literature, encompassing
theoretical and empirical reviews and addressing the existing research gaps. Moving on to Chapter 3, the research plan is delved into, encompassing aspects such as research design, philosophical considerations, variable operationalization, conceptual framework, empirical model, and data sources. Likewise, Chapter Four is dedicated to the presentation of data analysis and subsequent discussions. Concluding
the report, the final chapter outlines the conclusions drawn, recommendations, and
potential avenues for future extensions.

