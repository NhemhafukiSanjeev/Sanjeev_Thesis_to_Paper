\clearpage

\begin{center}
	\section*{\large{CHAPTER II \\ \vspace{-0.3cm} REVIEW OF LITERATURE}}
\end{center}
\addcontentsline{toc}{section}{\textbf{CHAPTER II: REVIEW OF LITERATURE}}
\renewcommand{\thepage}{\arabic{page}}
\setstretch{1.5}
This chapter reviews theoretical and empirical literature, encompassing theoretical
issues and its empirical evidence. The scientific literature available in credible sources
and references is examined. Such as the Journals and Google Scholar.\\
\subsection*{2.1  Theoritical Review}
\addcontentsline{toc}{subsection}{2.1 Introduction}
\renewcommand{\thepage}{\arabic{page}}
\setstretch{1.5}
The main economic theory to study sustainable livelihoods was developed by Robert Chambers and Gordon Conway in mid 1980s. \cite{chambers1992sustainable} contends that a sustainable livelihood is one that can withstand stress and shock, maintain or improve its assets and capabilities, and create opportunities for future generations to live sustainably. It also generates benefits for other livelihoods both locally and globally as well as over the long term. The author created the Sustainable Livelihood Approach (SLA) for the purpose of evaluating various vulnerability contexts to improve the effectiveness of development cooperation. 

Based on the Sustainable Livelihood Approach (SLA), the Sustainable Livelihood Framework (SLF) was proposed with a particular emphasis on the institutional processes which mediate the ability to carry out combination of livelihood strategies with the given livelihood resources in a particular context to achieve an outcome. Some of the well-known livelihood frameworks are those proposed by Department of International Department \cite{dfid1999sustainable}, \cite{ellis1999rural} and \cite{scoones2013livelihoods}. 

\cite{dfid1999sustainable} defines livelihoods broadly and systematically, considering the various assets that individuals or communities can draw upon for sustainable living. It emphasizes the inter-linkage of various capitals (Human, Social, Financial, Physical, and Natural Capital) the households possess with the livelihood outcomes. The framework provides a holistic viewpoint by taking into consideration the dynamic exchange of the capitals and how they influence the result of livelihood. 

\cite{ellis1999rural}  builds on the DFID model by introducing the idea of vulnerability and emphasizes the importance of understanding the factors that makes certain people or groups more vulnerable to shocks and stresses. The author highlights the significance of understanding the elements that make people or communities more vulnerable to shocks and pressures and presents the idea of vulnerability as a major predictor of livelihood strategies and results.

\cite{scoones2013livelihoods} contributed to SLF issue by emphasizing the importance of social relations and political economy in determining the livelihoods. The author's work highlights the need for critical analysis of social relations and political context and how they impact the household's ability to secure more sustainable livelihoods. The framework stress the need for a comprehensive understanding of livelihoods, taking into consideration not only the assets and vulnerability but also the the social, economic and political dimension. 

In a nutshell, DFID framework provides a comprehensive overview of various capitals that impact livelihoods. It acknowledges the concept of vulnerability but doesn't make it a central focus. Also, it does not explicitly delve into social and political dimension of livelihoods. Ellis on the other side of the spectrum, introduces the concept of vulnerability and places a strong emphasis on understanding the elements that increase the likelihood of failure. The framework includes some consideration of political and institutional factors but is centered more on vulnerability. Scoone highlights the critical role of political economy and power structures in shaping the livelihoods. Each framework brings unique perspectives to the understanding of sustainable livelihoods, with varying levels of focus on capitals, vulnerability, political economy, and power dynamics.    

\subsection*{2.2 Household Vulnerability}
\addcontentsline{toc}{subsection}{2.2 Household Vulnerability}
\renewcommand{\thepage}{\arabic{page}}
\setstretch{1.5}
Vulnerability is a concept that is applied in various disciplines, including engineering, ecology, economics, psychology and sociology \citep{fang2016rural}. Vulnerability refers to a state in which a person feels insecure when something harmful occurs \citep{chambers2006vulnerability}. "Vulnerable" refers to something that is likely to be harmed or wounded in everyday language. The term "Vulnerable", which means "wound", is dervied from the Latin word "vulnerare," \citep{calvo2005measuring}. On a similar note, \cite{chambers1989editorial}  states that vulnerability “refers to exposure to contingencies and stress, which is defenselessness, meaning a lack of means to cope without damaging loss”. 

The concept of household vulnerability is both controversial and multifaceted \citep{zhang2020capital}. So, household vulnerability analysis requires identification of not only the threat, but also the ‘resilience’, or responsiveness, in exploiting opportunities, and in resisting, or recovering from, the negative consequences of a changing environment \citep{moser1998asset}. In this perspective, \cite{bernier2014resilience} defines the resilience as the ability of a person, household, community, or system to adapt over time to shocks and proactively lower the risk of future shocks is what we refer to as resilience; these efforts promote growth and development as opposed to stability.

Many scholars conceptualise resilience as capacities that are driven by a set of capitals to produce outcomes such as influencing preparedness, mitigating impacts, and enhancing recovery against some risks. \cite{gaisie2021complexity} reveals complex relationship between household capitals and disaster outcomes in Ghana. The study finds that household capitals indicating higher economic status were linked to worse impacts from flooding but were essential for facilitating household recovery over time. \cite{zhang2020capital} in the similar study conducted in China suggests that all forms of capital (financial, human, natural, physical, and social 
capital) of a household were important determinants of household vulnerability. The term "resilience" and "vulnerability" has been used in the literature as antonyms of one another. The basic concept is that the more resilient a system, the less vulnerable it is. 

\cite{fang2016rural}, by constructing a composite vulnerability index, assessed the household vulnerability of the households in Shigatze Prefecture in Tibet
Autonomous Region (TAR) in China. The index has been constructed by taking into account the factors such as food variability, literacy rate of labor force, cash income and expenditure, precipitation vulnerability and drought area. The study finds that the factors under consideration reflects the close relationship between the basic requirement of th rural households in te harsh plataeu environment, less developed regions and vulnerability.    

\cite{antwi2013characterising} assessed the vulnerability to drought across six communities in Ghana. The study reveals varying vulnerability degrees influenced by socioeconomic factors. Authors find that less vulnerable households depict the resilience through alternative livelihoods and social connections. On a similar study, \cite{rahman2023households} quantifies cyclone vulnerability in rural Bangladesh, emphasizing  the multidimensional nature of vulnerability encompassing social, economic, physical, institutional, environmental, and attitudinal factors. The research, focusing on Kalmegha and Patharghata regions, reveals distinct vulnerability patterns, particularly in environmental and composite aspects. \cite{notenbaert2013derivation} constructed the household vulnerability index and explores the vulnerability and coping capacity related to current variability conditions with  focus on the adaptive capacity of the households. The study suggests that distance to paved road, income diversification and savings of the households significantly influences the household vulnerability. 

\subsection*{2.3 Review of National Studies}
\addcontentsline{toc}{subsection}{2.3 Review of National Studies}
\renewcommand{\thepage}{\arabic{page}}
\setstretch{1.5}
Numerous studies have been conducted with respect to the assessment of vulnerability across Nepal. The following study uses the country-level data to investigate the vulnerability. \cite{aksha2019analysis} investigated the social vulnerability in Nepal by adapting Social Vulnerability Index (SoVI) methods to Nepalese context using the full data set of 2011 census provided by the Central Bureau of Statistics (CBS). The study employs the Principal Component Analysis (PCA) to generate the independent set of factors to calculate the SoVI score. The SoVI for Nepal was calculated for each spatial unit (3918 village development committee and 53 municipalities). The components used in the study are renters and occupation, poverty and poor infrastructure, favorable social conditions, migration and gender, ethnicity, medical services, education. The study finds that social vulnerability is particularly high in areas that have concentrations of Dalit and Minority populations.    

\cite{shahiestimating} estimate the vulnerability score for Nepal using a three-stage feasible generalized least square technique to assess vulnerability to poverty. Utilizing the third round of Nepal Living Standards Survey data, the study's findings reveal that Nepal's overall vulnerability is 33 percent, indicating the probability of households falling into poverty due to various shocks such as death, illness, unemployment, and other idiosyncratic factors. The vulnerability score is notably high for minority populations. The authors identify Karnali and Sudurpaschhim regions as having a higher proportion of highly vulnerable households 

On a household level, \cite{bista2019grasping} examines the relationship between the magnitude of climate variability and household vulnerability in the catchment areas of the Sot Khola sub-water basin in the western mountainous region of Surkhet, Nepal. The author constructs a theoretical climate vulnerability index based on household-level data collected from 642 households, covering adaptive capacity, sensitivity, and exposure. The findings reveal that a majority, 52.7\%, of households are sensitive to climate-induced disasters such as landslides and floods due to their socio-economic status and food insufficiency. The study suggests that, overall, 67\% of households are vulnerable to varying degrees, ranging from moderate to extremely high vulnerability, while the remaining 33\% are least vulnerable.


Another household survey study by \cite{mainali2019mapping} employs a mixed-method approach, utilizing the Livelihood Vulnerability Index (LVI) at the community level. By integrating data from over 900 household surveys and national-level databases, the authors map the climate vulnerability of ten drought-prone villages in the central-east mid-hill region of Nepal. The findings reveal significant spatial variation in vulnerability, even within the lowest administrative units. Livelihood strategies, water availability, and topographic factors were among the key determinants of vulnerability, with strong interconnections among these components. 

A study by \cite{gerlitz2017multidimensional} based on Hindu Kush Himalayas (HKH) collects data from 2311 households from six districts (Khotang, Udaypur, Siraha, Dolakha, Sunsari, Kavrepalanchok) in Koshi sub-basin in Nepal and computes the Multi-dimensional Livelihood Index (MLVI). The MLVI was constructed using  AF method (Alkire \& Foster, 2011). Several variables as an indicator of Adaptive capacity, Sensitivity and Exposure has been employed to form a composite MLVI. The author finds, among the six districts, Khotang showed the highest multidimensional livelihood vulnerability with 96\% of the population were multidimensionally vulnerable to change and on average vulnerable in regard to 52\% of the 25 vulnerability indicators, resulting in an index value of 0.50.  Udayapur district showed the highest absolute contribution of lack of adaptive capacity to livelihood vulnerability 0.16.         

\subsection*{2.4 Research Gap}
\addcontentsline{toc}{subsection}{2.4 Research Gap}
\renewcommand{\thepage}{\arabic{page}}
Theoretical literature suggests that household vulnerability is influenced by a variety of factors, including income, assets,  idiosyncratic and co-variate shocks. However, there has been limited research conducted in Nepal that has examined this issue using both cross-sectional and panel datasets.  

Household vulnerability analysis in Nepal's rural areas play a pivotal role in understanding the socio-economic dynamics of the population, particularly in the face of various shocks and stressors. However, it is essential  comprehending the inter-temporal dynamics of vulnerability. This gap is particularly pronounced in the context of rural households across the diverse physiographic regions of Nepal. Several studies in Nepal have examined the household vulnerability on a national level by employing census data from Central Bureau of Statistics (CBS). \cite{aksha2019analysis, shahiestimating} are the studies done on a national level. Household level analysis also have been done employing household surveys. \cite{bista2019grasping, mainali2019mapping, gerlitz2017multidimensional} carried out the vulnerability analysis using cross-sectional surveys. 

Existing studies on household vulnerability in Nepal predominantly rely on cross-sectional data, providing a snapshot of vulnerability at a specific point in time. The temporal dimension is crucial in unraveling the nuanced changes in vulnerability over time. A dearth of studies employing panel data sets hinders our ability to capture the trajectory of vulnerability and identify patterns of persistence. Employing panel data sets is imperative to unravel the inter-temporal dynamics of household vulnerability. Such datasets allow for the tracking of individual households over time, enabling researchers to discern patterns of vulnerability persistence, identify key determinants, and assess the effectiveness of interventions.

Nepal's diverse physio-graphic regions contribute to significant variations in socioeconomic and environmental conditions. Yet, a substantial literature gap exists in incorporating a comprehensive geographic perspective into household vulnerability analysis. The absence of studies encompassing survey data from rural households across different physio-graphic regions impedes our understanding of regional disparities and specific vulnerabilities unique to each area. The physio-graphic diversity in Nepal implies that vulnerabilities and coping mechanisms may differ across regions. A comprehensive understanding of household vulnerability necessitates survey data from rural households in each physio-graphic region. This approach would unveil region-specific challenges, enabling targeted policy recommendations.

In conclusion, addressing the identified literature gap requires a twofold approach: the utilization of panel data sets to capture the inter-temporal dynamics of household vulnerability and the collection of survey data from rural households in the diverse physio-graphic regions in Nepal. Bridging these gaps is crucial for advancing our understanding of household vulnerability, informing evidence-based policies, and ultimately enhancing the resilience of rural communities in Nepal.

\clearpage