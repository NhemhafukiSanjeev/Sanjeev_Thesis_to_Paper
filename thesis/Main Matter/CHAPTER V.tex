\clearpage
\begin{center}
	\section*{\large{CHAPTER V \\ \vspace{-0.3cm} SUMMARY AND CONCLUSIONS}}
\end{center}
\addcontentsline{toc}{section}{\textbf{CHAPTER V:  CONCLUSION AND RECOMMENDATIONS }}
\renewcommand{\thepage}{\arabic{page}}
\setstretch{1.5}
In this chapter, the summary of the study is presented in the first section. The conclusion, recommendations, and potential extensions are shown in the subsequent
section. \\
\subsection*{5.1 Summary of the Findings}
\addcontentsline{toc}{subsection}{5.1 Summary of the Findings}
\renewcommand{\thepage}{\arabic{page}}
\setstretch{1.5}
The central focus of this thesis is the comprehensive analysis of household vulnerability in rural Nepal. The research aims to delve into the inter-temporal dynamics of vulnerability and conduct a physio-graphic analysis by constructing a household vulnerability index (HVI). The HVI is developed by integrating various capitals, including Human capital, Physical capital, Financial capital, Livelihood strategies, and Social capital. Additionally, the study explores the factors influencing household vulnerability in the context of rural Nepalese households.

The research employs a unique and environmental augmented household-level livelihood panel data-set, as outlined in \citep{walelign2022unique}. This data-set spans the period from 2006 to 2012 and is sourced from Tribhuvan University’s Institute of Forestry and the University of Copenhagen’s Department of Food and Resource Economics. The data-set serves as a valuable resource for capturing the nuances of household livelihoods over time. 

The construction of the household vulnerability Index is a pivotal aspect of the research methodology, accomplished through the application of Mini-max and Maxi-min methods. These methods enable a comprehensive assessment of vulnerability by considering multiple dimensions. The analysis of the household vulnerability reveals distinct patterns. Chitwan exhibits a slightly variable vulnerability level with a mean household vulnerability index (HVI) of 0.62 in 2006, and 0.61 in 2009 and 2012. Kaski shows stable vulenrability, of 0.62 across 2006, 2009 and 2012. Mustang indicates higher vulnerability in 2006 (0.64), a slight decrease in 2009 (0.63), and stability in 2012.  Radar charts further depict the variability within districts, with Kaski's polygon suggesting consistent vulnerability, Chitwan's indicating at potential decrease, and Mustang's showing fluctuations followed by stabilization.  

Furthermore, the study employs Panel Data Regression techniques, including Pooled Ordinary Least Squares (OLS), Fixed Effects (FE), and Random Effects (RE) regression, to identify and quantify the factors influencing household vulnerability. Notably, environmental dependence emerges as a key contributor to increased household vulnerability in the rural households.

The study reveals a positive association between dependency ratios and the household vulnerability Index (HVI), indicating that a higher dependency ratio exacerbates vulnerability. Additionally, the analysis suggests that shocks to households also play a substantial role in elevating vulnerability levels. The research finds a mitigating effect of debt on household vulnerability, suggesting that households with certain debt levels exhibit lower levels of vulnerability. 


\subsection*{5.2 Conclusion}
\addcontentsline{toc}{subsection}{5.2 Conclusion}
\renewcommand{\thepage}{\arabic{page}}
\setstretch{1.5}
The research examines the relationship between the household vulnerability and environmental dependence in the rural households across the physio-graphic regions utilizing various capital such as Human Capital, Physical Capital, Social Capital, Financial Capital and Livelihood. A composite index was developed using the mini-max normalization method. The index allowed the analysis of the vulnerability on a household, village and district level. 

From the analysis, we found that Chitwan district had a stable vulnerability across all waves of the survey (2006, 2009, 2012) with a mean HVI of 0.61. Kaski district exhibited mild variability, with a decrease in mean HVI from 2006 to 2009 and an increase back to the original level in 2012.
Mustang district showed relatively higher household vulnerability in 2006, a slight decrease in 2009, and stability in 2012. 

After the household vulnerability analysis, we investigated its relationship with Environmental Dependence. The household vulnerability index, calculated based on established equations, served as the dependent variable. Environmental dependence as a major determinant, consistently exhibited a significantly positive association with vulnerability. If the dependence persists further, it might have an adverse effect on the households as well as the environment. Household vulnerability increases because they are prone to environmental risk. From the environmental perspective, this dependence is a threat to sustainable environment as it will degrade the environment.

Additionally, the study attempted to study the association of Debt on vulnerability. Debt, as a coping mechanism, displayed a negative relationship with vulnerability, though statistically insignificant. Debt plays important role of reducing vulnerability of the households in multi-faceted manner. It acts as a safety nets during the times of shocks and crisis by providing immediate financial support. It also enables households to invest in productive activities which reduces vulnerability and increases resilience capacity of the households in the long-run. Further, it helps to strengthen the capitals of the households that make the household resilient. Capitals such as Financial and Social Capital.  

A high dependency ratio is characterized by a large proportion of dependents relative to the working-age population. A high dependency ratio has a positive effect on household vulnerability. Having a high dependency ratio leads to heightened vulnerability to economic shocks and poverty, perpetual cycles of financial instability and deprivation. Having higher number of elderly and children means that the adult working age family member needs to refrain from participating in the labor force in order to do the care work for the dependents. This in turn exacerbates the vulnerability of the households. 

Households in rural Nepal face a lot of shocks that impact their livelihoods severely. Natural disasters, economic downturns, or health shocks or any other idiosyncratic or co-variate shocks increase the vulnerability of the household. Moreover, shocks exacerbate pre-existing vulnerabilities, disproportionately affecting poor and vulnerable. Shocks have a multifaceted influence on the household and vulnerability. The result of the study confirms that households facing more number of shocks are more vulnerable. 

\subsection*{5.3 Recommendations and Possible Extensions}
\addcontentsline{toc}{subsection}{5.3 Recommendations  and Possible Extensions}
\renewcommand{\thepage}{\arabic{page}}
\setstretch{1.5}
This thesis builds national and sub-national level vulnerability assessment \cite{antwi2013characterising, aksha2019analysis, shahiestimating} by developing and applying a household vulnerability index to characterize the nature and inter-temporal dynamics of the household vulnerability across the distinct geographic regions of Nepal. This study targets an important gap in the literature, improving understanding of the processes and factors that affect vulnerability, with a view to guiding the development of effective policies. The findings and result has shown that across the distinct geographical setting, different communities and households may experience differential vulnerability that may be attributed to differences in capitals possessed by the households. The analysis showed that across the rural setting communities, Mountainous village of Mustang district had a relatively higher household vulnerability. 

This contrast between the Nepal's diverse geographic landscape, significantly influences rural livelihoods and prevalence of subsistence economies. These differences must be taken into account when drafting the interventions to address specific vulnerabilities in areas exhibiting different patterns. Further, it is important to prioritize the environmental factors and household's dependency on environment. This thesis finds that there is a positive relationship between Environmental dependency and household vulnerability. So, policies that attempt to lower the dependency on the environment must be designed and implemented in rural setting. Additionally, community-level resilience programs, educational initiatives, and cross-sectoral collaboration could be proposed to address the multifaceted nature of household vulnerability.  These programs and policies may help to foster sustainable development, resilience, and improved livelihoods in rural Nepalese communities while accounting for the specific geographical and economic characteristics of each region. 

Further, there is a need to leverage debt to reduce the vulnerability. The households should be facilitate to invest the funds from the debt in productive assets, education, healthcare, and social capital to enhance resilience and income-generating opportunities. Alongside, household needs to practice responsible borrowing and effective debt management strategies to prevent excessive indebtedness and financial instability. Policies should be aimed at enhancing access to credit and financial services for the vulnerable households to facilitate economic empowerment and resilience-building initiatives.

In part of dependency ratio, there needs to be a social safety nets to provide services to the dependent population so that the financial strain on household with higher dependency ratio could be alleviated to some extent. Further, care facilities needs to be available in communities so that the working age population could participate in the labor force, without worrying about the dependent family members in the house. 

Also to mitigate the vulnerability to shocks, policies should be aimed at providing adequate safety nets and designing welfare programs to provide immediate relief and support to the vulnerable households affected by the shocks. Government needs to investments in disaster preparedness, risk reduction, and resilient infrastructure. This will ensure quick relief as well as the recovery.   

Moreover, there is a possibility for this version of the study to be expanded. Due to time and resource constraints, we used the equal weighting approach at the time of computing the household vulnerability index. In future studies, consultation with stakeholders regarding the weight of the the components of the household vulnerability could be discussed and apply the weights.

